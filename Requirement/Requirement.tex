% ======================================================================

% Author: Jiaye William Wang
% Creation Time: 2019-04-03 00:09:29
% Last Modified: 2019-04-05 13:49:48

\documentclass{article}

% Packages --------------------------------------------------------------------
\usepackage[bottom=1in, left=1in, right=1in, top=1in]{geometry}
\usepackage{amsfonts, amsmath, amssymb}
\usepackage[none]{hyphenat}
\usepackage[dvipsnames]{xcolor}

% Setups ----------------------------------------------------------------------
\title{}
\author{}
\date{}

% Start -----------------------------------------------------------------------
\begin{document}

\section{Introduction}
	\subsection{Purpose}
	The purpose of this document is to provide a detailed description of how 
	to use Selenium to solve an online $9 \times 9$ standard Sudoku puzzle 
	from Web Sudoku.
	\\[0.5cm]
	It will be a useful reference for people who want to write a Sodoku 
	solver software or how to use Selenium with Python3, or both. On the 
	other hand, it also allows puzzle solvers to check their work or 
	provided answers directly if they got stuck. 
	
	\subsection{Scope}
	This software will not include any Sudoku solving-strategies, instead 
	of using computer science algorithm, backtracking, which will directly 
	output the result of the puzzle.  

	\subsection{Definitions}
	\begin{center}
	\begin{tabular}{ | l | l | }
	\hline
	Term 	 & Definition 						\\
	\hline
	Selenium & The portable framework for fetch information from web script. \\
	\hline
	WebDriver& An API for Selenium 					\\
	\hline
	API 	 & A set of subroutine definitions, communication protocols, 
		and tools for building software				\\
	\hline
	\end{tabular}
	\end{center}
	\subsection{References}
	\begin{itemize}
		\item[1] IEEE Software Engineering Standards Committee, ``IEEE
			Std 830-1998, IEEE Recommended Practice for Software
			Requirements Specifications'', October 20, 1998
	\end{itemize}
	\subsection{Overview}

\section{Overall Description}
	\subsection{Product Perspective}
	\subsection{Product functions}
	\subsection{User Characteristics}
	\subsection{Design and Implementation Constraints}
	\subsection{Assumptions and Dependencies}

\section{Specific Requirements}
	\subsection{External Interfaces}
	\subsection{Functions}
	\subsection{Performance requirements}
	\subsection{Communications Interfaces}

\section{Appendixes}


\end{document}

